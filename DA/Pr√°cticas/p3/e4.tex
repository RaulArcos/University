Para las pruebas de caja negra se ha utilizado el algoritmo voraz de la P1, básicamente rellenamos la estructura \textit{Cell} con los valores de cada celda por defecto y llamamos al procedimienton de ordenación correspondiente, después de eso utilizamos la función voraz de la P1 para colocar la mejor celda en el mapa y se mide el tiempo del proceso completo.
\\\\
Se utiliza a continuación como ejemplo la prueba de caja negra de la ordenación rápida.
\begin{lstlisting}
//TERCER METODO: ORDENACION RAPIDA.
    cronometro c3;
    long int r3 = 0;
    c3.activar();
    do{
        CellRapida(cell, 0, N-1);

       while(currentDefense != defenses.end()) {
            pos = voraz(freeCells,cell,nCellsWidth, nCellsHeight, mapWidth, mapHeight, obstacles, defenses, (*currentDefense)->radio,(*currentDefense)->id,N);
            (*currentDefense)->position.x = pos.x;
            (*currentDefense)->position.y = pos.y;
            (*currentDefense)->position.z = 0; 
            ++currentDefense;
        }	 
        ++r3;
    }while(c3.tiempo() < e_abs / e_rel + e_abs);
    c3.parar();
\end{lstlisting}