
\begin{itemize}
    \item \textbf{Sin Ordenación}, consta de dos "for" que recorren la función, lo que nos lleva a \textbf{O(n\^2)}.
    \item \textbf{Ordenación por fusión}, Dividimos el problema en dos iguales, esto está representado por "log n", y el número de pasos es "log(n+1)" en el peor de los casos, por otro lado para encontrar el punto medio de la matriz necesitamos un paso O(1), para las submatrices será de O(n), por lo que la complejidad final será de \textbf{O(n*log n)}.
    \item \textbf{Ordenación rápida}, en el mejor caso, el pivote termina en el centro de la lista, en ese caso será de O(n*log n), en cambio, en el peor de los casos el pivote termina en un extremo de la lista, O(n\^2), por lo que el promedio es: \textbf{O(n·log n)}.
    \item \textbf{Ordenación por montículos}, El algoritmo es de orden n en ambas funciones, lo que resulta en \textbf{O(n)}
\end{itemize}
\pagebreak