Una vez colocado el centro de extracción, el valor otorgado al resto de defensas 
irá enfocado a rodear el centro de extracción.
\\
En mi caso, como se comenta brevemente en el punto anterior, otorgo el valor a las celdas con la función \textit{cellValue}, que 
mediante una flag determina si estamos colocando el centro de extracción o una simple defensa, 
por lo que una vez tenemos el centro de extracción en la posición (en mi  opinión) óptima, lo rodeamos de 
defensas para alargar lo máximo posible el alcance de los UCOS a esta.
\\
La ecuación para determinar el valor es la siguiente: \textit{value = 1000 - (distancia al centro de extracción)}.
\\
He elegido este método debido a que al rodear al centro, hay menos huecos que no entren en el radio de ataque de las defensas por los que puedan pasar los UCOS.
