Para la resolución de este problema vamos a utilizar el algoritmo estrella visto en la asignatura.
\\
El código utilizado es una traducción a C++ del pseudocódigo enseñado en las diapositivas de la asignatura.
\\
Este algoritmo expande el arbol de forma ordenada, necesita una heurística, que en este caso es la distancia entre dos vértices (estimada).
\\
Las estructuras de datos utilizadas para el algoritmo son las siguientes:
\begin{itemize}
    \item \textbf{AStarNode}, utilizada tanto individualmente como en forma de vector de nodos, utilizada para representar el conjunto de nodos abiertos,cerrados y el nodo actual por separado.
\end{itemize}
El algoritmo funciona de manera que se mide la distancia de cada casilla del mapa respecto a la de las defensas del mapa de manera que los UCOS traten de ir directos a la defensa 0 esquivando el rango del resto de las defensas.
\\
Se realiza, tal y como visto en las diapositivas de la asignatura las siguientes operaciones:
\begin{itemize}
    \item \textbf{push()}, introducimos el elemento en el montículo.
    \item \textbf{pop()}, extraemos el tope del montículo.
    \item \textbf{update()}, ordenamos el montículo en el orden segun nuestra condicion (en nuestro caso que distancia de a 'mayor' distancia de b).
\end{itemize}
