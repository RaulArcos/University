La única función voraz con la que cuento ya fué incluida en el ejercicio 3, 
se puede observar que una vez pasada la factibilidad se pregunta si la id (de la defensa) es 0.
//
En ese mismo else, podemos ver que es el valor que se le otorgan a las celdas cuando se tiene en cuenta que son 
defensas, aquí hay un fragmento del código de \textit{cellValue} que indica el valor que otorgará a las celdas 
cuando estamos teniendo en cuenta que son para defensas:
\begin{lstlisting}
if(central)
        value = (1000 - distanceCellToCenter)/nearObstacle;
    else
    //En cambio para las defensas, buscamos que rodeen a la posición de la central de extracción.
        value = (1000 - distanceCellToDefenseZero);
    
    return value;
\end{lstlisting}